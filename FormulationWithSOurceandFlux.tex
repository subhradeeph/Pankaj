\documentclass{article}
\usepackage{amsfonts}
\usepackage[top=0.5in, bottom=1in, right=1in, left=1in]{geometry}
\usepackage{graphicx}
\usepackage{caption}
\usepackage{subcaption}
\usepackage{cite}
\usepackage{multirow}
%\usepackage{multicolumn}
\usepackage{hhline}
\usepackage{amsmath}
\usepackage{mathtools}
\usepackage{amssymb}
\usepackage[T1]{fontenc}
%\usepackage{tabto}

\begin{document}
\section{Cahn-Hilliard Equation with Source Term}
Bulk free energy density:
\begin{gather}\label{bfe}
  f(c,\phi) = A(c-0.5)^2(1-h_{\phi}) + B(c-0.1)^2(c-0.9)^2h_{\phi} + (1-\chi)P g_{\phi} \\
  where, h_{\phi} = \phi^3 (10-15\phi + 6\phi^2), \nonumber \\
  g_{\phi} = \phi^2(1-\phi)^2 \nonumber 
\end{gather}
Cahn-Hilliard equation with a source term at the surface is written as:
\begin{equation}\label{CHS}
  \begin{gathered}
    \frac{\partial c}{\partial t} = M \triangledown^2 \mu + S \phi (1-\phi) \\
    \frac{\partial c}{\partial t} = M \triangledown^2 \Big \{ \frac{\partial f}{\partial c} - 2 \kappa_c \triangledown^2 c\Big\} + S \phi (1-\phi)  
  \end{gathered}
\end{equation}
Apply Foruier Spectral Method to discretize spatial derivative of Cahn-Hilliard equation. The general relationship for it in Fourier space is given as
\begin{align} \label{sp_der}
  &\frac{\partial^n u}{\partial x^n} = ((\sqrt{-1})k)^n \{u\}_k \\
  &where, \text{\textit{k} is the coefficient \textit{k}th Fourier mode}. \nonumber
\end{align}
By taking forward difference for time derivative and utilizing Eq. \eqref{sp_der}, discretized form of Eq. \eqref{CHS} is written as:
\begin{subequations}\label{eq:subeq}
\begin{align}
  \frac{\tilde{c}^{t+\triangle t} - \tilde{c}^t}{dt} = -M \textit{k}^2 \tilde{\frac{\partial f}{\partial c}}^t - 2M\kappa_c        \textit{k}^4 \tilde{c}^{t + \triangle t} + S \tilde{\phi}^t(1- \tilde{\phi})^t \label{eq:subeq1}
\intertext{Rearrange} 
\tilde{c}^{t+\triangle t} = \frac{-M \textit{k}^2 \tilde{\frac{\partial f}{\partial c}}^t + S \tilde{\phi}^t(1- \tilde{\phi})^t}{1 + 2M dt\kappa_c  \textit{k}^4}\label{eq:subeq2}
\end{align}
\end{subequations}
Eq. \eqref{eq:subeq2} is the final discritized form of Cahn-Hilliard equation with a source term.

\section{Cahn-Hilliard Equation with Surface Flux}
Rewite the Cahn-Hilliard equation without the source term.
\begin{equation}\label{CH}
\frac{\partial c}{\partial t} = M \triangledown^2 \Big \{ \frac{\partial f}{\partial c} - 2 \kappa_c \triangledown^2 c\Big\}
\end{equation}
Further, dividing \eqref{CH} into two separate equations.
\begin{subequations}\label{eq:subeqn}
   \begin{align}
   \frac{\partial c}{\partial t} - M\triangledown^2 \mu = 0  \label{eq:subeqn1} \\
   \mu - \frac{\partial f}{\partial c} + \kappa  \triangledown^2  c = 0 \label{eq:subeqn2}
   \end{align}
\end{subequations}
Let $\eta$ and $\epsilon$ be the test functions for c and $\mu$ respectively and integrate Eq. \eqref{eq:subeqn1} and Eq. \eqref{eq:subeqn2}.

\begin{subequations}\label{eq:subeqnn}
   \begin{align}
     \int_{V} \left\{ \eta \left( \frac{\partial c}{\partial t} - M\triangledown^2 \mu \right)  \right\} dV = 0 \label{eq:subeqnn1} \\
     \int_{V} \left\{\epsilon \left( \mu - \frac{\partial f}{\partial c} + \kappa  \triangledown^2  c \right) \right\}dV = 0
     \label{eq:subeqnn2}
   \end{align}
\end{subequations}

Integrate by parts Eq. \eqref{eq:subeqnn1} and we get,
\begin{align}
\int_{V} \eta \frac{\partial c}{\partial t} dV + \int_{V} M \triangledown \eta . \triangledown \mu dV + \eta (-M \triangledown \mu) = 0
\end{align}
NBC: $M \triangledown \mu \equiv Q_1$ \\
EBC: c $\equiv Q_2 $ \\
Integrate Eq. \eqref{eq:subeqnn2} by parts
\begin{equation}
\int_{V} \epsilon \mu dV - \int_{V} \epsilon \frac{\partial f}{\partial c} dV - \int_{V} \kappa \triangledown \epsilon . \triangledown c dV +  \epsilon \left( \kappa \triangledown c \right)= 0
\end{equation}
NBC: $\kappa \triangledown c \equiv Q_3$ \\
EBC: $\mu \equiv Q_4 $ \\
So, Boundary conditions for our case can be taken as:\\
\begin{equation} \label{BC1}
\hat{n}	.(\kappa \triangledown c) = 0
\end{equation}
\begin{equation}\label{BC2}
\hat{n}.J = - R(x,c,\mu)
\end{equation}
\qquad where R is net reaction rate.\\
From Singh, Ceder and Bazant paper, R is equal to the difference of net insertion rate minus net exertion rate.\\
For our model, $R = R_{ins}$ and it is given by:
\begin{equation}\label{Rins}
R_{ins} = k_{ins}c_e exp\Big\{(\mu_e-\mu)/(\rho k_BT)\Big\}
\end{equation}
where, $\mu_e$ is the external chemical potential in a reservoir phase outside of the particle (eg, $Li^+$ in the electrolyte of $Li^+$ battery).\\
We know,
\begin{gather} 
\mu = \frac{\delta F}{\delta c} \\ 
\text{where}, F = f(c,\phi) - \frac{1}{2} \kappa_c (\triangledown c)^2 - \frac{1}{2} \kappa_{\phi} (\triangledown \phi)^2 \nonumber
\end{gather}
Utilizing \eqref{bfe} $\mu$ becomes
\begin{gather}\label{mu}
\mu = 2A(c-0.5)\hat{h_\phi} + 2B(c-0.1)(c-0.9)(2c-1)h_\phi - \chi P g_\phi - \kappa_c \triangledown^2c -\kappa_{\phi} \triangledown^2 \phi
\end{gather}
Substitue Eq. \eqref{mu} in \eqref{Rins}\\
Let $\beta = \frac{1}{ k_B T}$
\begin{equation}
\begin{multlined}
R = R_{ins} = k_{ins}exp \Big \{ \Big [ \mu_e - \Big( 2A(c-0.5)\hat{h_\phi} \\
    + 2B(c-0.1)(c-0.9)(2c-1) h_\phi -\chi P g_\phi - \kappa_c \triangledown^2 c - \kappa_{\phi} \triangledown^2 \phi \Big) \Big ]/  k_B T \Big \} \nonumber \\
    \end{multlined}
\end{equation}
\begin{equation}
R = k_{ins}exp\Big \{ \beta ( \mu_e - 2A(c-0.5)\hat{h_\phi} -2B(c-0.1)(c-0.9)(2c-1)h_\phi - \chi P g_\phi ) \Big \} exp\Big \{ \beta \kappa_c \triangledown^2c \Big\} exp\Big \{ \beta \kappa_{\phi} \triangledown^2 \phi \Big\} \nonumber
\end{equation}
\begin{gather}\label{final}
R = \bar{R} exp\Big \{ \beta \kappa_c \triangledown^2c \Big\} exp\Big \{ \beta \kappa_{\phi} \triangledown^2 \phi \Big\} \\
\text{where,} \bar{R} = exp\Big \{ \beta ( \mu_e - 2A(c-0.5)\hat{h_\phi} -2B(c-0.1)(c-0.9)(2c-1)h_\phi - \chi P g_\phi ) \Big \} \nonumber 
\end{gather}
Eq \eqref{CH} together with two natural boundary conditions \eqref{BC1}, \eqref{BC2} and \eqref{final} are the final form of equations. 
\end{document}
